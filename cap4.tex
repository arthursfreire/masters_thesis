\chapter{Descrição da Abordagem Proposta}
\label{descricao}

Os principais objetivos desta pesquisa são propor uma abordagem baseada em \textit{Redes Bayesianas} para auxiliar na avaliação da ETE, e um modelo para equipes baseadas em \textit{Scrum}, que se adeque à abordagem. Como descrito na Seção \ref{introducao:resultados}, a abordagem proposta é dividida em etapas que englobam desde a concepção do modelo baseado em \textit{Redes Bayesianas}, até a avaliação dos resultados calculados por ele e o processo de tomada de decisões por parte dos gerentes. Neste capítulo, serão descritas todas as etapas desse abordagem. A Figura \ref{} contém o fluxo do método completo e as interações entre as etapas.

{\color{red} Figura ilustrando o método...}

À princípio, na etapa de construção do modelo, deve ser feito um levantamento de quais fatores afetam a ETE no contexto em que o método será aplicado. Uma vez que esses fatores elencados, faz-se necessário construir o GAD. Em seguida, após a construção do GAD, é necessário definir as funções de probabilidade para cada um dos nós do GAD. Após a definição dessas funções, vem a etapa de avaliação do modelo. Posteriormente, deve-se alimentar os nós de entrada do modelo para que, assim, sejam calculados os resultados referentes à ETE. De posse desses resultados, o gerente deve fazer a análise deles, para que, em seguida, possam tomar decisões corretivas e preventivas relativas à ETE.
