\chapter{Estudo de Caso}
\label{estudodecaso}

Estudo de caso é uma metodologia de pesquisa adequada para estudar fenômenos contemporâneos em seu contexto natural \cite{runeson}. Com base nisso e na necessidade de avaliar o modelo proposto neste e sua utilização, foi realizado um estudo de caso no Laboratório de Sistemas Embarcados e Computação Pervasiva (Embedded Lab)\footnote{\url{http://www.embeddedlab.org/}}. O Embedded Lab está localizado na Universidade Federal de Campina Grande (UFCG) e foi escolhido em virtude da relações envolvendo a academia e a indústria.

Vários projetos são executados no Embedded Lab em parceria com empresas com o objetivo de desenvolver produtos de \textit{software}. Em todos os projetos do Embedded Lab com foco em desenvolvimento de \textit{software}, a metodologia para gestão e planejamento utilizada é o \textit{Scrum}. Portanto, o contexto no qual este estudo de caso foi realizado é o de indústria, com utilização de \textit{Scrum} como metodologia ágil adotada. Assim, os resultados e conclusões obtidos neste estudo de caso são referentes a esse contexto. O estudo de caso foi realizado em quatro projetos, onde cada um deles foi considerado uma unidade de análise. {\color{red} A duração foi de X dias.}

\section{\textit{Design} do Estudo de Caso}
\label{estudodecaso:design}

\subsection{Objetivos}
\label{estudodecaso:design:objetivos}

Para este estudo de caso, foram definidos dois principais objetivos:

\begin{enumerate}
  \item Verificar a fidelidade do modelo proposto para a avaliação do TE de equipes \textit{Scrum} com relação ao mundo real;
  \item Verificar a utilidade da abordagem para utilização do modelo em projetos \textit{Scrum}.
\end{enumerate}

\subsection{Objetos de Estudo}
\label{estudodecaso:design:objetos}

Os objetos de estudo são:

\begin{enumerate}
  \item O modelo proposto para representar o TE de equipes \textit{Scrum};
  \item A abordagem proposta para utilização do modelo.
\end{enumerate}

Logo, com base nos objetos de estudo definidos, deseja-se avaliar: a precisão do modelo proposto, a sua utilidade para auxiliar na liderança de equipes \textit{Scrum} e A facilidade de implementação e utilização da abordagem proposta.

\subsection{Questões de Pesquisa}
\label{estudodecaso:design:perguntas}

Com base nos objetivos definidos para este estudo de caso e visando alcançá-los, foram definidas as seguintes questões de pesquisa:

\begin{itemize}
  \item \textit{PP1}: O modelo proposto mensura de forma precisa o Trabalho em Equipe de equipes Scrum?
  \item \textit{PP2}: A utilização do modelo auxília na detecção de oportunidades de melhoria do Trabalho em Equipe de equipes Scrum?
  \item \textit{PP3}: A abordagem proposta é de fácil implementação e utilização?
  \item \textit{PP4}: O custo-benefício de utilizar a abordagem é positivo?
\end{itemize}

Dadas as questões de pesquisa definidas acima, as seguintes hipóteses foram definidas para respondê-las:

\begin{itemize}
  \item \textit{H0-1}: O modelo proposto não mensura de forma precisa o Trabalho em Equipe de equipes Scrum;
  \item \textit{HA-1}: O modelo proposto mensura de forma precisa o Trabalho em Equipe de equipes Scrum;
  \item \textit{H0-2}: A utilização do modelo não auxilia na detecção de oportunidades de melhoria do Trabalho em Equipe de equipes Scrum;
  \item \textit{HA-2}: A utilização do modelo auxilia na detecção de oportunidades de melhoria do Trabalho em Equipe de equipes Scrum;
  \item \textit{H0-3}: A abordagem proposta não é de fácil implementação e utilização;
  \item \textit{HA-3}: A abordagem proposta é de fácil implementação e utilização;
  \item \textit{H0-4}: O custo-benefício de utilizar a abordagem não é positivo;
  \item \textit{HA-4}: O custo-benefício de utilizar a abordagem é positivo.
\end{itemize}

Assim, \textit{H0-1} e \textit{HA-1} estão relacionadas à \textit{PP1}, \textit{H0-2} e \textit{HA-2} estão relacionadas à \textit{PP2}, \textit{H0-3} e \textit{HA-3} estão relacionadas à \textit{PP3}, e \textit{H0-4} e \textit{HA-4} estão relacionadas à \textit{PP4}.

\subsection{Unidades de Análise}
\label{estudodecaso:design:unidades}

{\color{red} Ainda não está definido... Precisamos das informações das equipes.}

\subsection{Sujeitos - Quem utiliza a abordagem e o modelo?}
\label{estudodecaso:design:sujeitos}

Para cada unidade de análise, os sujeitos são líderes de projeto que atuam como \textit{Scrum Masters}. No Embedded Lab, esses sujeitos realizam atividades relacionadas ao processo e o gerenciamento da equipe, atividades relacionadas ao \textit{design} dos produtos, do ponto de vista gráfico e arquitetural de produto, além de implementação.

{\color{red} Falta descrever os Scrum Masters em termos de experiência...}

\subsection{Métodos}
\label{estudodecaso:design:metodos}

A coleta de dados é uma atividade necessária para responder as questões de pesquisas de um estudo de caso experimental. De acordo com Lethbridge et al. \cite{}, há três diferentes categorias de métodos para coleta de dados: direto (e.g., entrevistas), indireto (e.g., \textit{survey}) e independente (e.g., análise de documentação). Portanto, o método utilizado para coleta de dados desse estudo de caso é o indireto, uma vez que os dados serão coletados por meio de questionários.

\section{Coleta dos Dados}
\label{estudodecaso:coleta}



\section{Análise dos Dados}
\label{estudodecaso:analise}



\section{Resultados}
\label{estudodecaso:resultados}

{\color{red} Os resultados serão obtidos após o final do estudo de caso...}
