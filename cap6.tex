\chapter{Estudo de Caso}
\label{estudodecaso}

Estudo de caso é uma metodologia de pesquisa adequada para estudar fenômenos contemporâneos em seu contexto natural \cite{runeson}. Com base nessa afirmação e na necessidade de avaliar o modelo proposto neste trabalho e sua utilização, foi realizado um estudo de caso no Laboratório de Sistemas Embarcados e Computação Pervasiva (Embedded Lab)\footnote{\url{http://www.embeddedlab.org/}}. O Embedded Lab está localizado na UFCG e foi escolhido em virtude das suas relações envolvendo a academia e a indústria.

Vários projetos são executados no Embedded Lab em parceria com empresas com o objetivo de desenvolver produtos de \textit{software}. Em todos os projetos do Embedded Lab com foco em desenvolvimento de \textit{software}, a metodologia para gestão e planejamento utilizada é o \textit{Scrum}. Portanto, o contexto no qual este estudo de caso foi realizado é o de indústria, com utilização de \textit{Scrum} como metodologia ágil adotada. Assim, os resultados e conclusões obtidos neste estudo de caso são referentes a esse contexto. O estudo de caso foi realizado em X projetos, onde cada um deles foi considerado uma unidade de análise. {\color{red} A duração foi de X dias.}

\section{\textit{Design} do Estudo de Caso}
\label{estudodecaso:design}

\subsection{Objetivos}
\label{estudodecaso:design:objetivos}

Para este estudo de caso, foram definidos dois principais objetivos:

\begin{enumerate}
  \item Verificar a fidelidade do modelo proposto para a avaliação do TE de equipes \textit{Scrum} com relação ao mundo real;
  \item Verificar a utilidade da abordagem para utilização do modelo em projetos \textit{Scrum}.
\end{enumerate}

\subsection{Objetos de Estudo}
\label{estudodecaso:design:objetos}

Os objetos de estudo são:

\begin{enumerate}
  \item O modelo proposto para representar o TE de equipes \textit{Scrum};
  \item A abordagem proposta para utilização do modelo.
\end{enumerate}

Logo, com base nos objetos de estudo definidos, deseja-se avaliar: a precisão do modelo proposto, a sua utilidade para auxiliar na liderança de equipes \textit{Scrum} e A facilidade de implementação e utilização da abordagem proposta.

\subsection{Questões de Pesquisa}
\label{estudodecaso:design:perguntas}

Com base nos objetivos definidos para este estudo de caso e visando alcançá-los, foram definidas as seguintes questões de pesquisa:

\begin{itemize}
  \item \textit{PP1}: O modelo proposto mensura de forma precisa o TE de equipes Scrum?
  \item \textit{PP2}: A utilização do modelo auxília na detecção de oportunidades de melhoria do TE de equipes Scrum?
  \item \textit{PP3}: A abordagem proposta é de fácil implementação e utilização?
  \item \textit{PP4}: O custo-benefício de utilizar a abordagem é positivo?
\end{itemize}

Dadas as questões de pesquisa definidas acima, as seguintes hipóteses foram definidas para respondê-las:

\begin{itemize}
  \item \textit{H0-1}: O modelo proposto não mensura de forma precisa o Trabalho em Equipe de equipes Scrum;
  \item \textit{HA-1}: O modelo proposto mensura de forma precisa o Trabalho em Equipe de equipes Scrum;
  \item \textit{H0-2}: A utilização do modelo não auxilia na detecção de oportunidades de melhoria do Trabalho em Equipe de equipes Scrum;
  \item \textit{HA-2}: A utilização do modelo auxilia na detecção de oportunidades de melhoria do Trabalho em Equipe de equipes Scrum;
  \item \textit{H0-3}: A abordagem proposta não é de fácil implementação e utilização;
  \item \textit{HA-3}: A abordagem proposta é de fácil implementação e utilização;
  \item \textit{H0-4}: O custo-benefício de utilizar a abordagem não é positivo;
  \item \textit{HA-4}: O custo-benefício de utilizar a abordagem é positivo.
\end{itemize}

Assim, \textit{H0-1} e \textit{HA-1} estão relacionadas à \textit{PP1}, \textit{H0-2} e \textit{HA-2} estão relacionadas à \textit{PP2}, \textit{H0-3} e \textit{HA-3} estão relacionadas à \textit{PP3}, e \textit{H0-4} e \textit{HA-4} estão relacionadas à \textit{PP4}.

\subsection{Unidades de Análise}
\label{estudodecaso:design:unidades}

\begin{longtable}{
|p{0.25\dimexpr \textwidth-3\arrayrulewidth-4\tabcolsep\relax}|
 p{0.25\dimexpr \textwidth-3\arrayrulewidth-4\tabcolsep\relax}|
 p{0.25\dimexpr \textwidth-3\arrayrulewidth-4\tabcolsep\relax}|
 p{0.25\dimexpr \textwidth-3\arrayrulewidth-4\tabcolsep\relax}|
}
\caption{Descrição das Unidades de Análise.\label{unidades}}\\

\hline
& \multicolumn{3}{|c|}{\textbf{Projeto}}\\
\hline
\multicolumn{1}{|c|}{\textbf{Característica}} & \textbf{A} & \textbf{B} & \textbf{C}\\
\hline
\endfirsthead

\hline
\multicolumn{4}{|c|}{Continuação da Tabela \ref{unidades}}\\
\hline
& \multicolumn{3}{|c|}{\textbf{Projeto}}\\
\hline
\multicolumn{1}{|c|}{\textbf{Característica}} & \textbf{A} & \textbf{B} & \textbf{C}\\
\hline
\endhead

\hline
\endfoot

\hline
% \multicolumn{2}{| c |}{End of Table}\\
% \hline\hline
% \endlastfoot

Experiência, em média de anos, dos integrantes da equipe participando em projetos de desenvolvimento de \textit{software} & 2.5 & 2 & 2 \\ \hline
Experiência, em média de anos, dos integrantes da equipe trabalhando em equipes ágeis & 1 & 1 & 2 \\ \hline
Breve descrição do cronograma do projeto & O cronograma inicial do projeto parecia ser tranquilo, mas como parte da equipe foi alocada para dar manutenção ao projeto anterior, acabou ficando mais apertado. & O projeto atual tem um cronograma que é relativamente fácil de alcançar, mas dependemos muito de outra entidade que está desenvolvendo o \textit{Hardware} que iremos trabalhar. & Cronograma sendo seguido no prazo, ficando apertado em alguns momentos. Algumas mudanças de requisitos geraram algum retrabalho, o que pode vir a comprometer algum item do cronograma inicial. \\ \hline
Breve descrição do escopo do projeto & Moderado. Os aplicativos a desenvolver são relativamente simples, mas a diversidade das plataformas suportadas contribui muito para a complexidade. & Uma ferramenta para monitoramento e controle de ativos de segurança patrimonial, é complexo. & Projeto de grande relevância para o cliente e complexo do ponto de vista de integração entre os componentes. O produto final depende da integração de componentes de \textit{software} e \textit{hardware} desenvolvidos por equipes do Embedded e do cliente. \\ \hline
Breve descrição da plataforma e tecnologias utilizadas no projeto & Plataforma: Desktop/Tablet (x86), Windows 8-10. Tecnologias: Visual Studio (com ReSharper), .NET / C\# e NUnit. & A plataforma utilizada no projeto é Android. Tecnologias utilizadas: SIP e RTSP. & Plataforma: Web. Tecnologias: Django e Python. \\ \hline

\end{longtable}

\subsection{Sujeitos - Quem utiliza a abordagem e o modelo?}
\label{estudodecaso:design:sujeitos}

Para cada unidade de análise, os sujeitos são líderes de projeto que atuam como \textit{Scrum Masters}. No Embedded Lab, esses sujeitos realizam atividades relacionadas ao processo e o gerenciamento da equipe, atividades relacionadas ao \textit{design} dos produtos, do ponto de vista gráfico e arquitetural de produto, além de implementação.

Na Tabela \ref{sujeitos}, são apresentados os perfis dos sujeitos em relação à experiência, em anos, desenvolvendo \textit{software}, liderando projetos de desenvolvimento, utilizando métricas no suporte à tomada de decisões e utilizando métodos ágeis.

\begin{table}[ht!]
\centering
\caption{Perfis dos Sujeitos}
\label{sujeitos}
\resizebox{\textwidth}{!}{\begin{tabular}{|l|c|c|c|}
\hline
 & \multicolumn{3}{c|}{\textbf{Sujeito}} \\ \hline
\multicolumn{1}{|c|}{\textbf{Característica}}                                           & \textbf{1}  & \textbf{2} & \textbf{3} \\ \hline
Experiência, em anos, trabalhando em projetos de desenvolvimento de \textit{software}            & 5           & 10         & 10         \\ \hline
Experiência, em anos, liderando projetos de desenvolvimento de \textit{software}                 & 0.5         & 3          & 2          \\ \hline
Experiência, em anos, utilizando métricas e indicadores no suporte à tomada de decisões & 1.5         & 2          & 6          \\ \hline
Experiência, em anos, utilizando métodos ágeis                                          & 5           & 2          & 7          \\ \hline
\end{tabular}}
\end{table}

\subsection{Métodos}
\label{estudodecaso:design:metodos}

A coleta de dados é uma atividade necessária para responder as questões de pesquisas de um estudo de caso experimental. De acordo com Lethbridge et al. \cite{lethbridge}, há três diferentes categorias de métodos para coleta de dados: direto (e.g., entrevistas), indireto (e.g., \textit{survey}) e independente (e.g., análise de documentação). Portanto, o método utilizado para coleta de dados desse estudo de caso é o indireto, uma vez que os dados serão coletados por meio de questionários.

\subsection{Procedimento}
\label{estudodecaso:design:procedimento}

Neste estudo de caso, foi utilizada a ferramenta AgenaRisk\footnote{\url{http://www.agenarisk.com/}} para efetuar os cálculos do modelo. Em virtude de algumas limitações com licensas da ferramenta, o modelo foi criado e todos os cálculos realizados na máquina do pesquisador. Após a obtenção dos resultados, eles foram apresentados aos sujeitos em seguida pelo pesquisador. Após a definição do modelo, e dos questionários para avaliação do TE e da abordagem, este estudo de caso foi dividido em duas fases: \textit{Treinamento} e \textit{Utilização da Abordagem}.

\subsubsection{Fase 1 - Treinamento}
\label{estudodecaso:design:procedimento:treinamento}

O objetivo desta fase do estudo de caso é prover aos sujeitos o entendimento dos conceitos relacionados aos objetos de estudo. Assim, espera-se que ao final dessa fase qualquer dúvida em relação à esses fatores seja sanada para que os resultados não sejam influenciados por má-interpretação das perguntas dos questionários.

À princípio, os conceitos de \textit{Redes Bayesianas}, \textit{Ranked Nodes}, além de \textit{Funções de Probabilidade}, suas aplicações e funcionamento foram explicados para facilitar o entendimento da construção do modelo. Após isso, o modelo proposto nesta dissertação, e o relacionamento entre os fatores que o compõem foram explicados. Em seguida, foi explicado como seria realizado o processo de coleta de dados e quais perguntas do questionário de alimentação do modelo são referentes à quais nós de entrada do modelo. Por fim, foi explicado como é feita a análise dos resultados gerados pelo modelo, e como é possível identificar oportunidades de melhoria no TE. Alguns exemplos foram utilizados nessa fase para auxiliar no entendimento dos sujeitos.

\subsubsection{Fase 2 - Utilização da Abordagem}
\label{estudodecaso:design:procedimento:abordagem}



\subsection{Ameaças à Validade}
\label{estudodecaso:design:ameacas}

Runeson et al. \cite{runeson} afirmam que há diferentes maneiras de classificar aspectos da validade e ameaças à validade na literatura. No trabalho anteriormente citado, eles definem um esquema de classificação que distingue bem quatro aspectos da validade de um estudo de caso. São eles: \textit{Validade de Construção}, \textit{Validade Interna}, \textit{Validade Externa e Confiabilidade}.

O aspecto da \textit{Validade de Construção} está relacionado com o fato de o que é estudado realmente representar o que o pesquisador tem em mente estar de acordo com as questões de pesquisa. Por exemplo, o assunto abordado nas entrevistas é interpretado pelos pesquisador e os entrevistados da maneira diferente. Portanto, neste estudo de caso, apesar do treinamento realizado para os sujeitos envolvidos, há a possibilidade deles interpretarem as perguntas dos questionários de tal forma que não condiz com os objetivos para os quais elas foram elaboradas.

A \textit{Validade Interna} diz respeito ao ato de verificar se um determinado fator afeta o fator investigado, quando há o risco de um terceiro fator que influenciar o fator investigado. Logo, como neste estudo de caso adotou-se o TE como indicador do desempenho da equipe, e há outros fatores como FATOR A, FATOR B e FATOR C que influenciam o desempenho da equipe \cite{}, também há ameaças à \textit{Validade Interna deste estudo}.

Com relação ao aspecto da \textit{Validade Externa}, que está relacionado em saber até que ponto é possível generalizar os resultados, e em que medida os resultados são de interesse para outras pessoas fora do caso investigado. Durante a análise da \textit{Validade Externa}, o pesquisador precisa analisar se os resultados podem ser relevantes para outros casos. Portanto, como os objetos de estudo deste estudo de caso foram avaliados para apenas X unidades de análise, talvez não seja possível generalizar os resultados para todas as equipes ágeis do mundo.

Além desses aspectos, também há a \textit{Confiabilidade}, que está relacionada à dependência dos dados coletados e sua análise em relação ao pesquisador. Assim, como é necessário que os sujeitos deste estudo de caso respondam questionários com o intuito de poder avaliar as equipes que estão sendo lideradas por eles, há o risco de haver viés nos dados coletados. Isso pode acontecer não apenas pelo fato dos sujeitos estarem envolvidos com suas equipes e o seu trabalho, mas também pela possibilidade dos questionários não serem claros o suficiente para facilitar a sua resposta. Além disso, este estudo de caso foi conduzido apenas com equipes \textit{Scrum}, uma dentre as várias metodologias ágeis existentes. Logo, esses fatores também afetam a \textit{Confiabilidade} deste estudo de caso.

\section{Coleta dos Dados}
\label{estudodecaso:coleta}

A coleta de dados necessária para responder as perguntas de pesquisa deste estudo de caso foi feita com a utilização de questionários, no formato de formulários online. Dessa forma, os sujeitos podem respondê-los quando acharem cômodo, de modo que não venha a incomodar em sua rotina de trabalho. Para a criação desses questionários, foi decidido utilizar o Google Forms\footnote{\url{https://www.google.com/forms/about/}}, ferramenta que permite criar questionários e armazenar os dados coletados neles em planilhas providas pela ferramenta Google Sheets\footnote{\url{https://www.google.com/sheets/about/}}. Além de permitir criar os questionários e armazenar os resultados, essas ferramentas também facilitam o compartilhamento de ambos, com a utilização de links.

Como forma de alimentar os nós de entrada do modelo, foi criado um questionário com perguntas simples e diretas, visando diminuir o tempo necessário para respondê-lo. Para cada nó de entrada do modelo, uma ou mais perguntas foram elaboradas, e suas respostas são todas objetivas, de única escolha, na seguinta escala: Verdadeiro, Mais Verdadeiro que Falso, Nem Verdadeiro nem Falso, Mais Falso que Verdadeiro, Falso, Não aplicável. Essa escala foi adotada com base na ferramenta Comparative Agility\footnote{\url{https://comparativeagility.com/}}, que mede o quão ágil uma organização/equipe é, pois acredita-se que ela se adequa bem à este caso, uma vez que há uma seção relacionada ao Trabalho em Equipe no \textit{survey} que essa ferramenta utiliza para coletar os dados. Além dos dados para alimentação dos nós, perguntas relacionadas às métricas para o cálculo da medida de desempenho das equipes também foram inseridas nesse questionário.

O questionário referente ao auxílio do modelo na tomada de decisões por parte dos sujeitos, contém perguntas diretas, que seguirão o mesmo padrão supracitado. Contudo, também haverá a oportunidade de inserção de texto puro, onde os sujeitos poderão comentar e dar mais opiniões à respeito da pergunta de pesquisa tratada. Essa mesma estratégia foi adotada para avaliar a facilidade da implementação e utilização da abordagem proposta, como também o custo-benefício de sua utilização.

{\color{red} Falta mapear as hipóteses com as questões dos questionários. Será feito quando os questionários estiverem completamente definidos.}

\section{Análise dos Dados}
\label{estudodecaso:analise}

\subsubsection{\textit{PP1}: O modelo proposto mensura de forma precisa o TE de equipes Scrum?}

{\color{red} Descrever como foi respondida essa pergunta}

\subsubsection{\textit{PP2}: A utilização do modelo auxília na detecção de oportunidades de melhoria do TE de equipes Scrum?}

Para responder essa pergunta, foi necessário avaliar as hipóteses \textit{H0-2} e \textit{HA-2}. Assim, como a pergunta X do questionário de satisfação, que é a mesma pergunta que \textit{PP2}, e ela é respondida utilizando uma escala \textit{Likert} de cinco pontos, foi definida a seguinte condição:

Caso $v_{q1B} \le 3$, onde $v_{q1B}$ representa a média das respostas para \textit{PP2}, deve-se aceitar \textit{H0-2}. Caso contrário, rejeita-se \textit{H0-2} e, consequentemente, assume-se que \textit{HA-2} é verdadeira.

\subsubsection{\textit{PP3}: A abordagem proposta é de fácil implementação e utilização?}

É necessário avaliar as hipóteses \textit{H0-3} e \textit{HA-3} para responder essa pergunta. Assim, como a pergunta X do questionário de satisfação, que é a mesma pergunta que \textit{PP3}, e ela é respondida utilizando uma escala \textit{Likert} de cinco pontos, foi definida a seguinte condição:

Caso $v_{q2B} \le 3$, onde $v_{q2B}$ representa a média das respostas para \textit{PP3}, deve-se aceitar \textit{H0-3}. Caso contrário, rejeita-se \textit{H0-3} e, consequentemente, assume-se que \textit{HA-3} é verdadeira.

\subsubsection{\textit{PP4}: O custo-benefício de utilizar a abordagem é positivo?}

De forma análoga à \textit{PP2} e \textit{PP3}, para responder essa pergunta, é necessário avaliar as hipóteses \textit{H0-4} e \textit{HA-4}. Também foi definida uma pergunta no questionário de satisfação que corresponde à essa (Pergunta X), e que é respondida utilizando uma escala \textit{Likert} de cinco pontos. Logo, também de forma análoga à \textit{PP2} e \textit{PP3}, foi definida a seguinte condição:

Caso $v_{q3B} \le 3$, onde $v_{q3B}$ representa a média das respostas para \textit{PP4}, deve-se aceitar \textit{H0-4}. Caso contrário, rejeita-se \textit{H0-4} e, consequentemente, assume-se que \textit{HA-4} é verdadeira.

\section{Resultados}
\label{estudodecaso:resultados}

{\color{red} Os resultados serão obtidos após o final do estudo de caso...}
