\chapter{Questionários para Alimentação do Modelo}
\label{questionarios}

Neste Apêndice estão as perguntas elaboradas para facilitar o processo de alimentação do modelo proposto. Cada pergunta possui cinco respostas possíveis, e apenas uma delas pode ser utilizada para responder as perguntas.

A Tabela \ref{questionarios:comunicacao} corresponde às perguntas definidas para os nós de entrada \textit{Distribuição da Equipe} e Meio de \textit{Comunicação}, que influenciam a \textit{Comunicação} da equipe. Na Tabela \ref{questionario:reunioes} estão as perguntas definidas referentes aos nós de entrada \textit{Monitoramento} e \textit{Presença de Todos os Membros}, que influenciam a qualidade das \textit{Reuniões Diárias}. A Tabela \ref{questionarios:orientacao}, por sua vez, corresponde às perguntas relacionadas aos nós de entrada \textit{Atributos Pessoais} e \textit{Expertise}, que influenciam a qualidade da \textit{Orientação da Equipe}. Apesar do nó \textit{Auto-Gerenciamento} depender dos valores dos nós \textit{Expertise}, \textit{Liderança Compartilhada} e \textit{Aprendizagem da Equipe}, a Tabela \ref{questionarios:autogerenciamento} só contém as perguntas referentes a esses dois últimos, pois a pergunta para o nó \textit{Expertise} foi definida na Tabela \ref{questionarios:orientacao}. Finalmente, na Tabela \ref{questionarios:autonomia}, está definida a pergunta referente ao nó de entrada \textit{Autonomia da Equipe}, que influencia diretamente na qualidade do \textit{Trabalho em Equipe}.

De acordo com a Seção \ref{modelo:gad}, todos os nós do modelo são \textit{Nós Ranqueados} com cinco estados (i.e., Muito Baixo, Baixo, Médio, Alto e Muito Alto). Assim, como as perguntas definidas neste Apêndice possuem cinco respostas possíveis, é possível mapeá-las para um estado possível de um determinado nó da seguinte maneira:

\begin{itemize}
  \item \textit{Falso}$\,\to\,$\textit{Muito Baixo}
  \item \textit{Mais Falso que Verdadeiro}$\,\to\,$\textit{Baixo}
  \item \textit{Nem Verdadeiro nem Falso}$\,\to\,$\textit{Médio}
  \item \textit{Mais Verdadeiro que Falso}$\,\to\,$\textit{Alto}
  \item \textit{Verdadeiro}$\,\to\,$\textit{Muito Alto}
\end{itemize}

Contudo, a pergunte referente ao nó \textit{Autonomia da Equipe} deve ser interpretada de maneira inversa, mas seguindo a mesma lógica:

\begin{itemize}
  \item \textit{Falso}$\,\to\,$\textit{Muito Alto}
  \item \textit{Mais Falso que Verdadeiro}$\,\to\,$\textit{Alto}
  \item \textit{Nem Verdadeiro nem Falso}$\,\to\,$\textit{Médio}
  \item \textit{Mais Verdadeiro que Falso}$\,\to\,$\textit{Baixo}
  \item \textit{Verdadeiro}$\,\to\,$\textit{Muito Baixo}
\end{itemize}



\begin{table}[ht!]
\centering
\caption{Perguntas referentes à qualidade da Comunicação da equipe.}
\label{questionarios:comunicacao}
\resizebox{\textwidth}{!}{\begin{tabular}{|c|c|l|}
\hline
\multicolumn{3}{|c|}{\textbf{Comunicação}}                                                                                                                                                                                                                                                                                                                                                                                                                                                        \\ \hline
\textbf{ID} & \textbf{Nó Correspondente} & \multicolumn{1}{c|}{\textbf{Pergunta}}                                                                                                                                                                                                                                                                                                                                                                                                                 \\ \hline
1           & Distribuição da Equipe     & \begin{tabular}[c]{@{}l@{}}\textbf{Os membros da Equipe de Desenvolvimento compartilham sempre o mesmo local de trabalho?}\\ \\ Verdadeiro - Todos os membros da equipe compartilham o mesmo local de trabalho.\\ Falso - Os membros da equipe não compartilham o mesmo local de trabalho.\\ \\ Opção 1: Falso\\ Opção 2: Mais Falso que Verdadeiro\\ Opção 3: Nem Verdadeiro nem Falso\\ Opção 4: Mais Verdadeiro que Falso\\ Opção 5: Verdadeiro\end{tabular} \\ \hline
2           & Meio de Comunicação        & \begin{tabular}[c]{@{}l@{}}\textbf{Os membros da Equipe de Desenvolvimento conversam cara-a-cara sempre que possível?}\\ \\ Verdadeiro - Os membros da equipe comunicam-se sempre cara-a-cara.\\ Falso - Os membros da equipe não se comunicam cara-a-cara.\\ \\ Opção 1: Falso\\ Opção 2: Mais Falso que Verdadeiro\\ Opção 3: Nem Verdadeiro nem Falso\\ Opção 4: Mais Verdadeiro que Falso\\ Opção 5: Verdadeiro\end{tabular}                                \\ \hline
\end{tabular}}
\end{table}



\begin{table}[ht!]
\centering
\caption{Perguntas referentes à qualidade das Reuniões Diárias da equipe.}
\label{questionario:reunioes}
\resizebox{\textwidth}{!}{\begin{tabular}{|c|c|l|}
\hline
\multicolumn{3}{|c|}{\textbf{Reuniões Diárias}}                                                                                                                                                                                                                                                                                                                                                                                                                                                                                                                                                                                                                                                                                                                \\ \hline
\textbf{ID} & \textbf{Nó Correspondente}                                              & \multicolumn{1}{c|}{\textbf{Pergunta}}                                                                                                                                                                                                                                                                                                                                                                                                                                                                                                                                                                                                                                 \\ \hline
3           & Monitoramento                                                           & \begin{tabular}[c]{@{}l@{}}\textbf{Os membros da equipe externam suas dificuldades e seu progresso em relação às atividades realizadas} \\ \textbf{de forma clara e objetiva?}\\ \\ Verdadeiro - Os membros da equipe externam suas dificuldades e seu progresso em relação às atividades \\ realizadas de forma clara e objetiva. \\ Falso - Os membros da equipe não relatam de forma clara as atividades nas quais estão envolvidos, ou \\ aproveitam a oportunidade para justificar decisões que foram tomadas.\\ \\ Opção 1: Falso\\ Opção 2: Mais Falso que Verdadeiro\\ Opção 3: Nem Verdadeiro nem Falso\\ Opção 4: Mais Verdadeiro que Falso\\ Opção 5: Verdadeiro\end{tabular} \\ \hline
4           & \begin{tabular}[c]{@{}c@{}}Presença de Todos \\ os Membros\end{tabular} & \begin{tabular}[c]{@{}l@{}}\textbf{Todos os membros da equipe estiveram presente durante as reuniões diárias?}\\ \\ Verdadeiro - Todos os membros da equipe estiveram presente durante as reuniões diárias.\\ Falso - Em nenhuma das reuniões diárias todos os membros estavam presentes.\\ \\ Opção 1: Falso\\ Opção 2: Mais Falso que Verdadeiro\\ Opção 3: Nem Verdadeiro nem Falso\\ Opção 4: Mais Verdadeiro que Falso\\ Opção 5: Verdadeiro\end{tabular}                                                                                                                                                                                                               \\ \hline
\end{tabular}}
\end{table}



\begin{table}[ht!]
\centering
\caption{Perguntas referentes à qualidade da Orientação da equipe.}
\label{questionarios:orientacao}
\resizebox{\textwidth}{!}{\begin{tabular}{|c|c|l|}
\hline
\multicolumn{3}{|c|}{\textbf{Orientação da Equipe}}                                                                                                                                                                                                                                                                                                                                                                                                                                                                                                                                                                                                                                                   \\ \hline
\textbf{ID} & \textbf{Nó Correspondente} & \multicolumn{1}{c|}{\textbf{Pergunta}}                                                                                                                                                                                                                                                                                                                                                                                                                                                                                                                                                                                                                     \\ \hline
5           & Atributos Pessoais         & \begin{tabular}[c]{@{}l@{}}\textbf{Os membros da equipe se dão bem entre si?}\\ \\ Verdadeiro - A mistura de personalidades dos membros da equipe contribui para que eles se dêem bem entre si\\ Falso - A mistura de personalidades dos membros da equipe não contribui para que eles se dêem bem entre si.\\ \\ Opção 1: Falso\\ Opção 2: Mais Falso que Verdadeiro\\ Opção 3: Nem Verdadeiro nem Falso\\ Opção 4: Mais Verdadeiro que Falso\\ Opção 5: Verdadeiro\end{tabular}                                                                                                                                                                                   \\ \hline
6           & Expertise                  & \begin{tabular}[c]{@{}l@{}}\textbf{Os membros da equipe possuem todo o conhecimento necessário para o desenvolvimento das estórias} \\ \textbf{da sprint com interseção?}\\ \\ Verdadeiro - Os membros da equipe possuem todo o conhecimento necessário para o desenvolvimento das estórias \\ da sprint com capacidade de substituir uns aos outros na realização das tarefas.\\ Falso - Os membros da equipe não possuem todo o conhecimento necessário para o desenvolvimento da estórias da sprint.\\ \\ Opção 1: Falso\\ Opção 2: Mais Falso que Verdadeiro\\ Opção 3: Nem Verdadeiro nem Falso\\ Opção 4: Mais Verdadeiro que Falso\\ Opção 5: Verdadeiro\end{tabular} \\ \hline
\end{tabular}}
\end{table}



\begin{table}[ht!]
\centering
\caption{Perguntas referentes à capacidade de Auto-Gerenciamento da equipe.}
\label{questionarios:autogerenciamento}
\resizebox{\textwidth}{!}{\begin{tabular}{|c|c|l|}
\hline
\multicolumn{3}{|c|}{\textbf{Auto-Gerenciamento}}                                                                                                                                                                                                                                                                                                                                                                                                                                                                                                                    \\ \hline
\textbf{ID} & \textbf{Nó Correspondente} & \multicolumn{1}{c|}{\textbf{Pergunta}}                                                                                                                                                                                                                                                                                                                                                                                                                                                                                    \\ \hline
7           & Liderança Compartilhada    & \begin{tabular}[c]{@{}l@{}}\textbf{A autoridade na tomada de decisões e na liderança é compartilhada entre os membros da equipe?}\\ \\ Verdadeiro - A autoridade na tomada de decisões e na liderança é compartilhada entre os membros da equipe.\\ Falso - A autoridade na tomada de decisões e na liderança não é compartilhada entre os membros da equipe.\\ \\ Opção 1: Falso\\ Opção 2: Mais Falso que Verdadeiro\\ Opção 3: Nem Verdadeiro nem Falso\\ Opção 4: Mais Verdadeiro que Falso\\ Opção 5: Verdadeiro\end{tabular} \\ \hline
8           & Aprendizagem da Equipe     & \begin{tabular}[c]{@{}l@{}}\textbf{A equipe se adapta facilmente às mudanças que ocorrem durante o projeto?}\\ \\ Verdadeiro - A equipe se adapta facilmente às mudanças que ocorrem durante o projeto.\\ Falso - A equipe não tem capacidade de se adaptar às mudanças que ocorrem durante o projeto.\\ \\ Opção 1: Falso\\ Opção 2: Mais Falso que Verdadeiro\\ Opção 3: Nem Verdadeiro nem Falso\\ Opção 4: Mais Verdadeiro que Falso\\ Opção 5: Verdadeiro\end{tabular}                                                        \\ \hline
\end{tabular}}
\end{table}



\begin{table}[ht!]
\centering
\caption{Pergunta referente à Autonomia da equipe.}
\label{questionarios:autonomia}
\resizebox{\textwidth}{!}{\begin{tabular}{|c|c|l|}
\hline
\textbf{ID} & \textbf{Nó Correspondente} & \multicolumn{1}{c|}{\textbf{Pergunta}}                                                                                                                                                                                                                                                                                                                                                                                                                                                                                                                                            \\ \hline
9           & Autonomia da Equipe        & \begin{tabular}[c]{@{}l@{}}\textbf{Há um agente externo interferindo em como a equipe executa suas tarefas?}\\ \\ Verdadeiro - Há um agente externo que sempre interfere em como a equipe deve executar suas atividades.\\ Falso - Não há um agente externo interferindo em como a equipe executa suas tarefas. \\ O agente externo colabora com a equipe para definir o que será executado e apenas quando adequado.\\ \\ Opção 1: Falso\\ Opção 2: Mais Falso que Verdadeiro\\ Opção 3: Nem Verdadeiro nem Falso\\ Opção 4: Mais Verdadeiro que Falso\\ Opção 5: Verdadeiro\end{tabular} \\ \hline
\end{tabular}}
\end{table}


\chapter{Segundo apêndice}
