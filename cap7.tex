\chapter{Conclusão}
\label{conclusao}

Nesta dissertação, abordou-se a problemática e relevância relacionadas à avaliação da qualidade do TE em equipes ágeis. Além do fato de os fatores que influenciam a qualidade do TE serem subjetivos, se as relações entre eles não forem claras, a avaliação da qualidade do TE pode ser considerada uma atividade complexa.

Com o propósito de amenizar esse problema e, assim, garantir a melhoria contínua do TE e o aumento das chances de sucesso do projeto, neste trabalho, foi apresentado um modelo baseado em RB, e uma abordagem para utilizá-lo.

Tomando como base o modelo proposto em \cite{freire}, foi realizada uma revisão literária para elencar os faltores que influenciam a qualidade do TE. Dessa forma, foi possível evoluir esse modelo, garantindo que ele englobe fatores essenciais. Além de evoluir a estrutura do modelo, as TPN do modelo apresentado neste trabalho foram definidas utilizando uma abordagem mais robusta do que a utilizada em \cite{freire}.

Para utilizar o modelo, foi proposta uma abordagem que pode ser dividida em quatro macro-etapas: (i) avaliação do modelo, onde o indivíduo que deseja utilizá-lo deve verificar se o modelo está consistente com o cenário que será aplicado; (ii) alimentação do modelo, em que os nós de entrada do modelo são alimentados; (iii) análise dos dados, comparando os resultados do modelo com a eficiência da equipe, e levando em consideração fatores externos que podem afetar essa eficiência, além da elaboração de um plano corretivo e preventivo; e (iv) ações corretivas e preventivas, onde o plano elaborado na etapa anterior é colocado em prática.

A validação do modelo e da abordagem foi realizada por meio de estudo de caso utilizando três projetos do Embedded Lab como unidades de análise. Conforme discutido na Seção \ref{estudodecaso:resultados}, o modelo mostrou-se fiel em relação aos cenários em que foi utilizado para avaliar a qualidade do TE. Além disso, a utilização desse modelo possibilita aos seus usuários oportunidades de melhoria no TE, auxiliando na tomada de decisões. De acordo com os sujeitos que participaram do estudo de caso, o custo-benefício da utilização da abordagem no dia-a-dia do processo é positivo.

Em virtude da quantidade de unidades análise do estudo de caso ser baixa, não há como concluir que a abordagem e modelo propostos é útil para todos os projetos de desenvolvimento de \textit{software} que utilizam metodologias ágeis. Contudo, baseado na diversidade dos projetos desenvolvidos pelas unidades de análise e nos resultados do estudo de caso, acredita-se que os objetivos desta pesquisa foram atingidos.

\section{Limitações}
\label{conclusao:limitacoes}

Apesar de ser concluir que o modelo a abordagem propostos nesta pesquisa cumprem com o objetivos descritos na Seção \ref{introducao:objetivos}, existem alguns fatores que limitam as conclusões realizadas. Conforme mencionado anteriormente, o número de unidades de análise do estudo de caso não é ideal. Além disso, o estudo de caso durou apenas três \textit{sprints} (i.e., 45 dias) e todas as unidades de análise utilizavam \textit{Scrum}. Portanto, a quantidade de dados coletados pode ter sido baixa e a validação com equipes que adotam apenas \textit{Scrum} pode afetar as conclusões.

\section{Trabalhos Futuros}
\label{conclusao:trabalhos}

Como trabalhos futuros, pretende-se criar uma ferramenta \textit{Open Source} que permita construir e utilizar RB. Atualmente, o processo de construção e utilização das RB é feito com a ferramenta \textit{AgenaRisk}, que exige o pagamento de uma licença anual.

Espera-se também que outros pesquisadores sintam-se motivados para aplicar o modelo e abordagem propostos em uma quantidade maior de projetos, por um maior período de tempo. Isso poderia melhorar a confiança da pesquisa para, possivelmente, incorporar o modelo e a abordagem na em processos na indústria.

Além disso, o modelo proposto neste trabalho pode ser integrado como componente ao modelo proposto por Perkusich et al. \cite{perkusich2013} \cite{perkusich2014}, que é capaz de detectar problemas no processo de densenvolvimento de \textit{software} baseados em \textit{Scrum}.
