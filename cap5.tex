\chapter{Descrição da Abordagem Proposta}
\label{descricao}

Como apresentado na Seção \ref{introducao:objetivos:especificos}, dois dos principais objetivos desta pesquisa são: Propor um modelo baseado em \textit{Redes Bayesianas} para avaliar o TE de equipes ágeis e uma abordagem para utilizar o modelo proposto. Portanto, para utilizar o modelo proposto no Capítulo \ref{modelo}, neste capítulo será descrita a abordagem proposta.

Essa abordagem é dividida em etapas que englobam desde a coleta dos dados para alimentação do modelo, até o processo de tomada de decisões preventivas e corretivas por parte dos gerentes de projeto. Além disso, propõe-se que essa aborgadem deve ser utilizada antes da reunião de \textit{Retrospectiva da Iteração} para que, durante essa reunião, os gerentes de projeto possam reportar os resultados obtidos para a equipe, além de auxiliar na tomada de decisões para a próxima iteração. Portanto, neste capítulo, serão descritas todas as etapas dessa abordagem. A Figura \ref{} contém o fluxo completo da abordagem e as interações entre as etapas.

{\color{red} Figura ilustrando o método...}

A primeira etapa da abordagem diz respeito à coleta de dados e alimentação dos modelos. As perguntas necessárias para coletar os dados referentes aos nós de entrada do modelo estão estão definidas no Apêndice \ref{questionarios}. Com isso, caso uma empresa/organização deseje avaliar o TE de suas equipes, ou um gerente de projeto em particular deseje fazê-lo individualmente, basta responder as perguntas referentes aos nós de entrada para, assim, obter os resultados calculados pelo modelo.

A segunda etapa consiste da análise dos resultados calculados pelo modelo. Como o fator principal para o qual o modelo calcula os resultados, \textit{Trabalho em Equipe}, é subjetivo, há muita dificuldade para calcular esse fator de forma objetiva. Contudo, o TE é um fator que tem influência sobre outros fatores que podem ser medidos de forma objetiva (e.g., FATOR A, FATOR B, FATOR C) \cite{}. Portanto, para facilitar a análise dos dados, propõe-se que os indivíduos que utilizam esta abordagem adotem os resultados calculados pelo modelo como indicadores de uma determinada métrica que represente um desses fatores que são influenciados pelo TE. Dessa forma, em vez de analisar os resultados calculados comparando-os com resultados esperados pelos gerentes de projeto, a análise será menos sujeita a viés. Entretanto, é necessário atentar para o fato de que o TE, possivelmente, não é o único fator que influencia o fator com o qual ele está sendo comparado. Logo, para realizar a análise dos dados, talvez seja necessário fazer algumas presunções, que podem afetar a validade dessa análise.

Baseado nos resultados calculados pelo modelo e pelas análises realizadas pelos gerentes de projeto, há a possibilidade de tomar medidas preventivas e corretivas para serem aplicadas nas iterações seguintes, garantindo, assim, a melhoria do TE e do produto e do processo como um todo.

Propõe-se que os indivíduos que desejem utilizar essa abordagem realizem esses três passos do início ao final do projeto, sempre ao final das iterações, para avaliar o TE das equipes na iteração que se passou. Além disso, também há a possibilidade de utilização dessa abordagem pare realizar predições. Essa aplicação é útil para que os gerentes de projeto possam avaliar tomar decisões preventivas antes do começo de um determinado de projeto, baseando-se na equipe com a qual ele irá trabalhar. Dessa forma, é possível diminuir os riscos referentes às interações entre os integrantes da equipe. 