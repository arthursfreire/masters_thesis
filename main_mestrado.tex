%%%%%%%%%%%%%%%%%%%%%%%%%%%%%%%%%%%%%%%%%%%%%%%%%%%%%%%%%%%%%%%%%%%%%%%%%%%%%%%%
%%
%% Para utilizar ese modelo sao necessarios os seguintes arquivos:
%%
%% copin.cls
%% copin.sty
%% mestre.sty
%%
%%%%%%%%%%%%%%%%%%%%%%%%%%%%%%%%%%%%%%%%%%%%%%%%%%%%%%%%%%%%%%%%%%%%%%%%%%%%%%%%

\documentclass[a4paper,titlepage]{copin}
\usepackage[portuges,english]{babel}
\usepackage{copin,mestre,epsfig}
\usepackage{times}

%-------------------------- Para usar acentuacaoo em sistemas ISO8859-1 ------------------------------------
% Se estiver usando o Microsoft Windows ou linux com essa codificacao, descomente essa linhas abaixo
% e comente as linhas referentes ao UTF8
%\usepackage[latin1]{inputenc} % Usar acentuacao em sistemas ISO8859-1, comentar a linha com  \usepackage[utf8x]{inputenc}
%-----------------------------------------------------------------------------------------------------

%-------------------------- Para usar acentuacao em sistemas UTF8 ------------------------------------
% Para a maior parte das distribuicoes linux, usar a opcao utf8x (lembrar de comentar as linha referente a ISO8859-1 acima)
\usepackage{ucs}
%\usepackage[utf8x]{inputenc}
\usepackage[utf8]{inputenc}
\usepackage[T1]{fontenc}
%-----------------------------------------------------------------------------------------------------


\usepackage{fancyheadings}
\usepackage{graphicx}
\usepackage{longtable} %tabelas longas, para tabelas que ultrapassam uma pagina
%\input{psfig.sty}

% ----------------- Para inserir gráficos no formato .eps -----------------
\usepackage{epstopdf}

% ----------------- Para inserir URLs -----------------
\usepackage{url}

% ----------------- Para inserir texto colorido -----------------
\usepackage{color}

% ----------------- Para inserir codigo fonte de linguagens de programacao no documento -------------
\usepackage{listings}
\lstset{numbers=left,
stepnumber=1,
firstnumber=1,
%numberstyle=\tiny,
extendedchars=true,
breaklines=true,
frame=tb,
basicstyle=\footnotesize,
stringstyle=\ttfamily,
showstringspaces=false
}
\renewcommand{\lstlistingname}{Código Fonte}
\renewcommand{\lstlistlistingname}{Lista de Códigos Fonte}
% ---------------------------------------------------------------------------------------------------

\selectlanguage{portuges}
\sloppy



\begin{document}



%%%%%%%%%%%%%%%%%%%%%%%%%%%%%%%%%%%%%%%%%%%%%%%%%%%%%%%%%%%%%%%%%%%%%%%%%%%%%%%%
\Titulo{Título da Dissertação}
\Autor{Arthur Silva Freire}
\Data{dd/mm/aaaa}
\Area{Ciência da Computação}
\Pesquisa{Engenharia de \textit{Software}}
\Orientadores{Hyggo Oliveira de Almeida  \\
              Angelo Perkusich  \\
	         (Orientadores)}

\newpage
\cleardoublepage

\PaginadeRosto

\newpage
\cleardoublepage

%%%%%%%%%%%%%%%%%%%%%%%%%%%%%%%%%%%%%%%%%%%%%%%%%%%%%%%%%%%%%%%%%%%%%%%%%%%%%%%%
\begin{resumo}
Resumo aqui

\end{resumo}

\newpage
\cleardoublepage

%%%%%%%%%%%%%%%%%%%%%%%%%%%%%%%%%%%%%%%%%%%%%%%%%%%%%%%%%%%%%%%%%%%%%%%%%%%%%%%%
\begin{summary}
Abstract Here




\end{summary}

\newpage
\cleardoublepage

%%%%%%%%%%%%%%%%%%%%%%%%%%%%%%%%%%%%%%%%%%%%%%%%%%%%%%%%%%%%%%%%%%%%%%%%%%%%%%%%
\begin{agradecimentos}
Agradecimentos aqui
\end{agradecimentos}

\clearpage

%%%%%%%%%%%%%%%%%%%%%%%%%%%%%%%%%%%%%%%%%%%%%%%%%%%%%%%%%%%%%%%%%%%%%%%%%%%%%%%%
%% Definicao do cabecalho: secao do lado esquerdo e numero da pagina do lado direito
\pagestyle{fancy}
\addtolength{\headwidth}{\marginparsep}\addtolength{\headwidth}{\marginparwidth}\headwidth = \textwidth
\renewcommand{\chaptermark}[1]{\markboth{#1}{}}
\renewcommand{\sectionmark}[1]{\markright{\thesection\ #1}}\lhead[\fancyplain{}{\bfseries\thepage}]%
	     {\fancyplain{}{\emph{\rightmark}}}\rhead[\fancyplain{}{\bfseries\leftmark}]%
             {\fancyplain{}{\bfseries\thepage}}\cfoot{}

%%%%%%%%%%%%%%%%%%%%%%%%%%%%%%%%%%%%%%%%%%%%%%%%%%%%%%%%%%%%%%%%%%%%%%%%%%%%%%%%
\selectlanguage{portuges}

\Sumario
\ListadeSimbolos
\listoffigures
\listoftables
\lstlistoflistings %lista de codigos fonte - Para inserir a listagem de codigos fonte
\newpage
\cleardoublepage

\Introducao


%%%%%%%%%%%%%%%%%%%%%%%%%%%%%%%%%%%%%%%%%%%%%%%%%%%%%%%%%%%%%%%%%%%%%%%%%%%%%%%%
%
% Hifenizacao - Colocar lista de palavras que nao devem ser separadas e que
% nao estao no dicionario portugues.
% As palavras do dicionario portugues ja sao separadas corretamente pelo lateX
%
\hyphenation{ Hardware Software etc  }


%%%%%%%%%%%%%%%%%%%%%%%%%%%%%%%%%%%%%%%%%%%%%%%%%%%%%%%%%%%%%%%%%%%%%%%%%%%%%%%%
%% A partir daqui coloque seus capitulos. Sugere-se que eles sejam inseridos com o comando \input
%% Da seguinte maneira:
%%
%% \chapter{Introdução}
\label{intro}

De acordo com Emam et al. \cite{emam}, a porcentagem de projetos de TI que sucedem varia entre 46 e 55 porcento. De acordo com os autores, o sucesso de projetos de TI depende de cinco fatores: satisfação do cliente, orçamento, cronograma, qualidade do produto e produtividade da equipe. Além disso, para uma disciplina aplicada, esses números representam um alto índice de falhas.

Boehm et al. \cite{boehm} identificaram seis principais razões de falha em projetos de \textit{software}: requisitos imcompletos, ausência de envolvimento do cliente, falta de recursos, expectativas irrealistas, ausência de suporte executivo e mudança de requisitos e especificações. A ocorrência da maioria desses fatores se dá por conta de problemas na comunicação e interação entre desenvolvedores e \textit{stakeholders}. Uma das principais razões pelas quais as metodologias ágeis têm se tornado popular no contexto do desenvolvimento de \textit{software}, é a necessidade de focar na melhoria da colaboração entre desenvolvedores e \textit{stakeholders}, além de melhorar a velocidade de resposta com relação à mudança de requisitos.

No Manifesto Ágil \cite{manifesto}, há afirmações que projetos que utilizam métodos ágeis devem focar nos indivíduos e nas relações entre eles em vez de focar em processos e ferramentas. Além disso, como é esperado que as equipes ágeis sejam auto-organizáveis, é necessário que os membros da equipe colaborem entre si e adotem os conceitos de responsabilidade e compromisso com as atividades da equipe. De acordo com Bustamante et al. \cite{bustamante}, numa equipe ágil ideal, os membros da equipe compartilham o mesmo ambiente de trabalho e comunicam-se cara-a-cara diariamente. Lalsing et al. \cite{lalsing} afirmam que o gerente de projeto deve definir as relações entre os papéis para garantir a efetividade na coordenação da equipe e o controle do projeto. Nesse último trabalho, os autores também afirmam que indivíduos com diferentes personalidades, geralmente, devem trabalhar juntos para garantir uma equipe coesa.

A utilização de metodologias ágeis requer a adoção de uma série de práticas que aumentam as chances de sucesso do projeto, pois a adoção dessas práticas resolve a maioria dos problemas responsáveis por falhas em projetos de \textit{software}. Assim, uma vez que a saída de um processo de \textit{software} é o próprio \textit{software}, a qualidade do produto final é dependente de uma série de artefatos e fatores que compõem esse processo.

Chow et al. \cite{chow}, identificaram os três principais fatores que influenciam o sucesso de projetos de desenvolvimento de \textit{software} que utilizam métodos ágeis: estratégia de entrega, técnicas de engenharia de \textit{software} no contexto ágil e a capacidade do time. Esse último, de acordo com os autores, está relacionado com o ato de construir projetos em volta de indivíduos motivados. Tendo em vista que as equipes são consideradas os recursos mais valiosos de projetos que utilizam métodologias ágeis, e sua capacidade, como citado anteriormente, é um dos principais fatores que influenciam o sucesso desses projetos, faz-se necessário atentar para os aspectos que influenciam a eficiência dessas equipes.

Em algumas pesquisas sobre equipes de desenvolvimento de \textit{software}, foi identificado que a eficiência dessas equipes está relacionada a eficiência da coordenação do trabalho em equipe \cite{kraut} \cite{hoegl}. Logo, se a eficiência do trabalho em equipe está relacionada com a eficiência das equipes, que por sua vez, influenciam o sucesso de projetos de desenvolvimento de \textit{software}, pode-se afirmar que a eficiência do trabalho em equipe também está relacionada com o sucesso desses projetos. Assim, a avaliação e melhora contínua da eficiência do trabalho em equipe é importante para garantir boa qualidade do \textit{software} resultante de um processo, assim como o sucesso do projeto.

\section{Problemática}
\label{intro:prob}

Conforme supracitado, é importante avaliar e garantir a melhoria contínua da ETE, a adoção de um método que proporcione essas oportunidades aos gerentes de projeto agregará valor à esses projetos. Contudo, há diversos fatores que podem vir a influenciar essa eficiência \cite{}. Apesar do contexto em que o objeto de estudo foi avaliado neste trabalho ser restrito, algumas equipes podem decidir utilizar determinadas práticas que outras equipes não utilizam. Esse fato pode causar diferença entre os conjuntos de fatores que influenciam a ETE de diferentes equipes.

Além disso, os fatores que influenciam a ETE, são, em sua grande maioria, subjetivos. Dessa forma, a aplicação de um método de avaliação da ETE precisa minimizar o viés que pode ser introduzido por conta da subjetividade desses fatores, garantindo que os resultados sejam fiéis ao cenário no qual a avaliação foi realizada.

\section{Objetivos}
\label{intro:obj}

Considerando o que foi abordado na seções anteriores, o principal objetivo deste trabalho é mitigar os problemas descritos, principalmente na Seção \ref{intro:prob}, propondo uma abordagem para auxiliar os gerentes de projeto na avaliação da ETE de equipes baseadas em \textit{Scrum}. Além disso, a utilização dessa abordagem deve auxiliar na identificação de oportunidades de melhorias no trabalho em equipe.

Como forma de representar a ETE em função do relacionamento dos fatores que a influenciam, optou-se pelo uso de \textit{Redes Bayesianas}, uma vez que modelos probabilísticos dessa família são adequados para se modelar incerteza em um determinado domínio \cite{bengal}. Essa decisão foi tomada com o objetivo de diminuir a incerteza em relação à confiança nos resultados finais do modelo, tendo em vista que, como citado na Seção \ref{intro:prob}, a maioria dos fatores que influenciam a ETE são subjetivos.

\subsection{Objetivos Específicos}
\label{intro:obj:esp}

Para simplificar os objetivos descritos na Seção \ref{intro:obj}, podemos especificá-los da seguinte maneira:

\begin{enumerate}
  \item Propor uma abordagem baseada em \textit{Redes Bayesianas} para auxiliar na avaliação da ETE de equipes baseadas em \textit{Scrum};
  \item Proporcionar aos gerentes de projeto uma abordagem menos sensível à subjetividade na avaliação da ETE, que possa auxiliar na identificação de oportunidades de melhorias no trabalho em equipe;
  \item Aplicar a abordagem em projetos reais de desenvolvimento de \textit{software}, baseados em \textit{Scrum}, para avaliar sua utilidade e seu custo-benefício.
\end{enumerate}

\section{Contribuições e Resultados}
\label{intro:result}

A abordagem proposta neste trabalho é dividida em etapas que englobam desde a concepção do modelo baseado em \textit{Redes Bayesianas} até a avaliação dos resultados calculados por ele. A utilização de modelos dessa família permite calcular probabilisticamente o impacto de fatores menos complexos (nós de entrada e nós intermediários) que afetam a ETE. Tendo em mãos esses resultados, os gerentes poderão observar as saídas do modelo, que o auxiliarão na identificação de oportunidades de melhorias no trabalho em equipe.

Mesmo sabendo que o conjuntos de fatores que influenciam a ETE de determinadas equipes podem ser diferentes, neste trabalho é proposto um modelo genérico para representar a ETE no contexto de \textit{Scrum}. Dessa forma, os indivíduos que desejarem utilizar essa abordagem poderão adotá-lo como base para modificá-lo e, assim, obtenham uma representação mais fiel em relação ao cenário em que a equipe está inserida.

{\color{red} Os resultados serão inseridos ao final do estudo de caso...}

\section{Relevância}
\label{intro:rel}

A abordagem proposta é uma alternativa promissora para auxiliar no processo de tomada de decisões por parte dos gerentes de projeto. Os resultados calculados pelo modelo permitem que eles avaliem quais fatores merecem mais atenção caso mais de um fator esteja dimnuindo a ETE, e quais atitudes podem ser tomadas para evitar riscos, além de quais atitudes podem ser tomadas para melhorar a EFT. Essas oportunidades de auxílio na tomada de decisões por parte dos gerentes existem porque, com a utilização dessa abordagem, é possível interpretar fatores subjetivos de forma mais objetiva.

Como a abordagem proposta proporciona os benefícios supracitados, e sabendo da relação entre a EFT e a qualidade do produto de \textit{software} resultante do processo, e também o processo em si, a sua utilização proporciona um aumento nas chances de sucesso do projeto. Além disso, essa abordagem pode ser integrada com outras abordagens que utilizadas na avaliação do processo de \textit{software} como um todo.

\section{Estrutura da Dissertação}
\label{intro:estr}

Esta dissertação está organizada da seguinte forma:

...

%% \chapter{Fudamentação Teórica}
\label{fund}

\chapter{Introdução}
\label{intro}

De acordo com Emam et al. \cite{emam}, a porcentagem de projetos de TI que sucedem varia entre 46 e 55 porcento. De acordo com os autores, o sucesso de projetos de TI depende de cinco fatores: satisfação do cliente, orçamento, cronograma, qualidade do produto e produtividade da equipe. Além disso, para uma disciplina aplicada, esses números representam um alto índice de falhas.

Boehm et al. \cite{boehm} identificaram seis principais razões de falha em projetos de \textit{software}: requisitos imcompletos, ausência de envolvimento do cliente, falta de recursos, expectativas irrealistas, ausência de suporte executivo e mudança de requisitos e especificações. A ocorrência da maioria desses fatores se dá por conta de problemas na comunicação e interação entre desenvolvedores e \textit{stakeholders}. Uma das principais razões pelas quais as metodologias ágeis têm se tornado popular no contexto do desenvolvimento de \textit{software}, é a necessidade de focar na melhoria da colaboração entre desenvolvedores e \textit{stakeholders}, além de melhorar a velocidade de resposta com relação à mudança de requisitos.

No Manifesto Ágil \cite{manifesto}, há afirmações que projetos que utilizam métodos ágeis devem focar nos indivíduos e nas relações entre eles em vez de focar em processos e ferramentas. Além disso, como é esperado que as equipes ágeis sejam auto-organizáveis, é necessário que os membros da equipe colaborem entre si e adotem os conceitos de responsabilidade e compromisso com as atividades da equipe. De acordo com Bustamante et al. \cite{bustamante}, numa equipe ágil ideal, os membros da equipe compartilham o mesmo ambiente de trabalho e comunicam-se cara-a-cara diariamente. Lalsing et al. \cite{lalsing} afirmam que o gerente de projeto deve definir as relações entre os papéis para garantir a efetividade na coordenação da equipe e o controle do projeto. Nesse último trabalho, os autores também afirmam que indivíduos com diferentes personalidades, geralmente, devem trabalhar juntos para garantir uma equipe coesa.

A utilização de metodologias ágeis requer a adoção de uma série de práticas que aumentam as chances de sucesso do projeto, pois a adoção dessas práticas resolve a maioria dos problemas responsáveis por falhas em projetos de \textit{software}. Assim, uma vez que a saída de um processo de \textit{software} é o próprio \textit{software}, a qualidade do produto final é dependente de uma série de artefatos e fatores que compõem esse processo.

Chow et al. \cite{chow}, identificaram os três principais fatores que influenciam o sucesso de projetos de desenvolvimento de \textit{software} que utilizam métodos ágeis: estratégia de entrega, técnicas de engenharia de \textit{software} no contexto ágil e a capacidade do time. Esse último, de acordo com os autores, está relacionado com o ato de construir projetos em volta de indivíduos motivados. Tendo em vista que as equipes são consideradas os recursos mais valiosos de projetos que utilizam métodologias ágeis, e sua capacidade, como citado anteriormente, é um dos principais fatores que influenciam o sucesso desses projetos, faz-se necessário atentar para os aspectos que influenciam a eficiência dessas equipes.

Em algumas pesquisas sobre equipes de desenvolvimento de \textit{software}, foi identificado que a eficiência dessas equipes está relacionada a eficiência da coordenação do trabalho em equipe \cite{kraut} \cite{hoegl}. Logo, se a eficiência do trabalho em equipe está relacionada com a eficiência das equipes, que por sua vez, influenciam o sucesso de projetos de desenvolvimento de \textit{software}, pode-se afirmar que a eficiência do trabalho em equipe também está relacionada com o sucesso desses projetos. Assim, a avaliação e melhora contínua da eficiência do trabalho em equipe é importante para garantir boa qualidade do \textit{software} resultante de um processo, assim como o sucesso do projeto.

\section{Problemática}
\label{intro:prob}

Conforme supracitado, é importante avaliar e garantir a melhoria contínua da ETE, a adoção de um método que proporcione essas oportunidades aos gerentes de projeto agregará valor à esses projetos. Contudo, há diversos fatores que podem vir a influenciar essa eficiência \cite{}. Apesar do contexto em que o objeto de estudo foi avaliado neste trabalho ser restrito, algumas equipes podem decidir utilizar determinadas práticas que outras equipes não utilizam. Esse fato pode causar diferença entre os conjuntos de fatores que influenciam a ETE de diferentes equipes.

Além disso, os fatores que influenciam a ETE, são, em sua grande maioria, subjetivos. Dessa forma, a aplicação de um método de avaliação da ETE precisa minimizar o viés que pode ser introduzido por conta da subjetividade desses fatores, garantindo que os resultados sejam fiéis ao cenário no qual a avaliação foi realizada.

\section{Objetivos}
\label{intro:obj}

Considerando o que foi abordado na seções anteriores, o principal objetivo deste trabalho é mitigar os problemas descritos, principalmente na Seção \ref{intro:prob}, propondo uma abordagem para auxiliar os gerentes de projeto na avaliação da ETE de equipes baseadas em \textit{Scrum}. Além disso, a utilização dessa abordagem deve auxiliar na identificação de oportunidades de melhorias no trabalho em equipe.

Como forma de representar a ETE em função do relacionamento dos fatores que a influenciam, optou-se pelo uso de \textit{Redes Bayesianas}, uma vez que modelos probabilísticos dessa família são adequados para se modelar incerteza em um determinado domínio \cite{bengal}. Essa decisão foi tomada com o objetivo de diminuir a incerteza em relação à confiança nos resultados finais do modelo, tendo em vista que, como citado na Seção \ref{intro:prob}, a maioria dos fatores que influenciam a ETE são subjetivos.

\subsection{Objetivos Específicos}
\label{intro:obj:esp}

Para simplificar os objetivos descritos na Seção \ref{intro:obj}, podemos especificá-los da seguinte maneira:

\begin{enumerate}
  \item Propor uma abordagem baseada em \textit{Redes Bayesianas} para auxiliar na avaliação da ETE de equipes baseadas em \textit{Scrum};
  \item Proporcionar aos gerentes de projeto uma abordagem menos sensível à subjetividade na avaliação da ETE, que possa auxiliar na identificação de oportunidades de melhorias no trabalho em equipe;
  \item Aplicar a abordagem em projetos reais de desenvolvimento de \textit{software}, baseados em \textit{Scrum}, para avaliar sua utilidade e seu custo-benefício.
\end{enumerate}

\section{Contribuições e Resultados}
\label{intro:result}

A abordagem proposta neste trabalho é dividida em etapas que englobam desde a concepção do modelo baseado em \textit{Redes Bayesianas} até a avaliação dos resultados calculados por ele. A utilização de modelos dessa família permite calcular probabilisticamente o impacto de fatores menos complexos (nós de entrada e nós intermediários) que afetam a ETE. Tendo em mãos esses resultados, os gerentes poderão observar as saídas do modelo, que o auxiliarão na identificação de oportunidades de melhorias no trabalho em equipe.

Mesmo sabendo que o conjuntos de fatores que influenciam a ETE de determinadas equipes podem ser diferentes, neste trabalho é proposto um modelo genérico para representar a ETE no contexto de \textit{Scrum}. Dessa forma, os indivíduos que desejarem utilizar essa abordagem poderão adotá-lo como base para modificá-lo e, assim, obtenham uma representação mais fiel em relação ao cenário em que a equipe está inserida.

{\color{red} Os resultados serão inseridos ao final do estudo de caso...}

\section{Relevância}
\label{intro:rel}

A abordagem proposta é uma alternativa promissora para auxiliar no processo de tomada de decisões por parte dos gerentes de projeto. Os resultados calculados pelo modelo permitem que eles avaliem quais fatores merecem mais atenção caso mais de um fator esteja dimnuindo a ETE, e quais atitudes podem ser tomadas para evitar riscos, além de quais atitudes podem ser tomadas para melhorar a EFT. Essas oportunidades de auxílio na tomada de decisões por parte dos gerentes existem porque, com a utilização dessa abordagem, é possível interpretar fatores subjetivos de forma mais objetiva.

Como a abordagem proposta proporciona os benefícios supracitados, e sabendo da relação entre a EFT e a qualidade do produto de \textit{software} resultante do processo, e também o processo em si, a sua utilização proporciona um aumento nas chances de sucesso do projeto. Além disso, essa abordagem pode ser integrada com outras abordagens que utilizadas na avaliação do processo de \textit{software} como um todo.

\section{Estrutura da Dissertação}
\label{intro:estr}

Esta dissertação está organizada da seguinte forma:

...



%%%%%%%%%%%%%%%%%%%%%%%%%%%%%%%%%%%%%%%%%%%%%%%%%%%%%%%%%%%%%%%%%%%%%%%%%%%%%%%%
%% BIbliografia
%% Coloque suas referencias no arquivo ref.bib e descomente as proximas duas linhas

\bibliographystyle{plain} % estilo de bibliografia   plain,unsrt,alpha,abbrv.
\bibliography{refs} % arquivos com as entradas bib.

%%%%%%%%%%%%%%%%%%%%%%%%%%%%%%%%%%%%%%%%%%%%%%%%%%%%%%%%%%%%%%%%%%%%%%%%%%%%%%%%
%% Apendice
% Caso seja necessario algum apendice, descomente a proxima linha.

\appendix
\chapter{Primeiro apêndice}

\chapter{Segundo apêndice}

%%%%%%%%%%%%%%%%%%%%%%%%%%%%%%%%%%%%%%%%%%%%%%%%%%%%%%%%%%%%%%%%%%%%%%%%%%%%%%%%

\end{document} 