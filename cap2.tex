\chapter{Fudamentação Teórica}
\label{fundamentacao}

{\color{red} Introdução...}

\section{Metodologias Ágeis}
\label{fundamentacao:ageis}

{\color{red} Metodologias Ágeis...}

\subsection{Fatores-Chave do Trabalho em Equipe}
\label{fundamentacao:ageis:fatores}

Como forma de elencar os fatores que influenciam o TE de equipes ágeis, foi realizada uma Revisão Literária, adotando alguns aspectos de Revisões Sistemáticas, para garantir... Para isso, os motores de busca selecionados para servir de fonte de trabalhos foram: \textit{ACM}\footnote{\url{http://dl.acm.org/}}, \textit{IEEE}\footnote{\url{http://ieeexplore.ieee.org/Xplore/home.jsp}}, \textit{Scopus}\footnote{\url{http://www.scopus.com/}}, \textit{Science Direct}\footnote{\url{http://www.sciencedirect.com/}} e \textit{Google Scholar}\footnote{\url{https://scholar.google.com.br/}}. Em seguida, foram definidas as \textit{strings} de busca para cada um desses motores. Além disso, a ferramenta \textit{StArt}\footnote{\url{http://lapes.dc.ufscar.br/tools/start_tool}} para auxiliar a gerência das informações referentes aos trabalhos encontrados.

Durante a fase de seleção dos trabalhos relevantes, os \textit{abstracts} de todos os trabalhos recuperados pelos motoroes

\section{Redes Bayesianas}
\label{fundamentacao:redes}

{\color{red} Redes Bayesianas...}

\subsection{Construção de Redes Bayesianas}
\label{fundamentacao:redes:construcao}

{\color{red} Construção...}

\subsection{Nós Ranqueados}
\label{fundamentacao:nos}

{\color{red} Nós Ranqueados...}
